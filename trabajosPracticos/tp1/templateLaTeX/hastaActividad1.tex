\section{Actividad práctica I}
    \indent{Para el circuito de la figura se tiene que calcular las corrientes $i_1$, $i_2$ e $i_3$ en función de la corriente total $i_T$. Para esto se utilizará la ley de Ohm y la ley de Kirchhoff.}
    % \ref{fig:figura1}

    \begin{center}
    \begin{circuitikz}[raised voltages]
        \draw (0,0)
         to[battery, v=$V_ S$, invert, f>_=$I_T$] (0,3)
         to[R=$R_1$, f>_=$I_1$] (3,3)
         to[short, *-*] (3,3)
         to[R=$R_2$, f>_=$I_2$] (3,0) -- (0,0);

         \draw (3,3) -- (6,3)
         to[R=$R_3$, f>_=$I_3$] (6,0) -- (3,0)
         to[short, *-*] (3,0);
        
         \node at (0,3.2) {A};
         \node at (3,3.2) {B};
         \node at (0,-0.2) {C};
    \end{circuitikz}
    \end{center}

    \indent{Los Valores son los siguientes:}
    
    \begin{itemize}
        \item $V_S = 10V$
        \item $R_1 = 10k\Omega$
        \item $R_2 = 4.7k\Omega$
        \item $R_3 = 3.3k\Omega$
    \end{itemize}
    \subsection{Calculo de la corriente total}
    \sangria{} Para calcular la corriente total del circuito, se usará la Ley de Ohm:
    \ecuacion{i_T = \frac{v_T}{R_T}}
    Nuestro objetivo ahora es calcular el valor de $R_T$. El voltaje $v_T$ es el valor dado por la fuente.
    \begin{enumerate}
        \item Nótese que los resistores $R_2$ y $R_3$ son paralelos en el circuito: 
            \begin{center} 
                $R_{23} (R_2 || R_3) = \frac{1}{{\frac{1}{R_2}} + \frac{1}{R_3}}$ \\[10pt]
                $R_{23} = \frac{1}{\frac{1}{4,7k\Omega} + \frac{1}{3,3k\Omega}}$ \\[10pt]
                \resaltar{R_{23} = \frac{1551}{800} k\Omega \approx 1,938k\Omega}
            \end{center}

        \item Volviendo al circuito, puede notar que el resistor 1 está en serie con el resistor $R_{23}$ equivalente, por lo que:
        \begin{center}
            $R_T = R_1 + R_{23}$ \\[10pt]
            $R_T = 10k\Omega + \frac{1551}{800}k\Omega$ \\[10pt]
            \resaltar{R_T = \frac{9551}{800}k\Omega \approx 11,938k\Omega}
        \end{center}
        \item Teniendo la resistencia total del circuito, la corriente $i_T$ queda:
            \begin{center}
                $i_T = \frac{v_T}{R_T}$ \\[10pt]
                $i_T = \frac{10v}{\frac{9551}{800}\cdot 10^3 \Omega}$
            \end{center}
            \columnbreak{}
            \begin{center}
                \resaltar{i_T = \frac{8}{9551}A \approx 8,37 \cdot 10^{-4}A}
            \end{center}
    \end{enumerate}
    \subsection{Caídas de tensión}
    \sangria{} Las caídas de tensión en los resistores vienen dados por cómo están conectados al circuito.
    \begin{enumerate}
        \item $R_1$: \\[5pt]
        \sangria{} Nótese que el resistor 1 está en serie con el resto de componentes del circuito, por lo que la corriente que lo atraviesa es la corriente total del circuito. Su caída de tensión viene dada por la Ley de Ohm:
        \begin{center}
            $v_{R_1} = R_1 \cdot i_T$ \\[10pt] 
            $v_{R_1} = (10 \cdot 10^3 \Omega)(\frac{8}{9551}A)$ \\[10pt] 
        \resaltar{v_{R_1} = \frac{8\cdot10^4}{9551}V\approx 8,3760V}\end{center}

        \item $R_2$ y $R_3:$ \\
            \sangria{} Ambos resistores están en paralelo, por lo que sus tensiones son iguales. \\ \sangria{}Considerando la caída de tensión en el resistor 1, la tensión en estos es el remanente:
        \begin{center} 
            $v_{\text{$R_2$ \& $R_3$}} = v_S - v_{R_1}$ \\[10pt]
            $v_{\text{$R_2$ \& $R_3$}} = 10V - \frac{8\cdot 10^4}{9551}V$ \\[10pt]
            \resaltar{v_{\text{$R_2$ \& $R_3$}} = \frac{15510}{9551}V \approx 1,6239V}
        \end{center}
    \end{enumerate}
    \subsection{Corrientes en $R_2$ y $R_3$}
    \sangria{} Aplicando la Ley de Kirchhoff de las corrientes, se puede calcular las corrientes faltantes. \\ \sangria{} Tomando de referencia el nodo B:
    \begin{center}
        \begin{circuitikz}
            \draw(0,0) node[fill=black, circle, inner sep=1.5pt] (N) {};
            \draw[-Stealth] (-2,0) -- (N) node[midway, above] {$i_T$};
            \draw[-Stealth] (N) -- (2,0) node[midway, above] {$i_3$};
            \draw[-Stealth] (N) -- (0,-2) node[midway, right] {$i_2$};
            \node at (0,0.3) {B};
        \end{circuitikz}
    \end{center}
    \noindent
    $
    \begin{cases}
        i_T - i_2 - i_3 = 0 (1) \\[5pt]
        v_{R_2} = i_2R_2 (2) \\[5pt]
    \end{cases}
    $
    \begin{center}
        $(2): v_{R_2}  = i_2R_2 \Rightarrow i_2 = \frac{v_{R_2}}{R_2} $ \\[10pt]
        $i_2 = \frac{\frac{15510}{9551}V}{4,7 \cdot 10^3 \Omega}$ \\[10pt]
        \resaltar{i_2 = \frac{33}{95510}A\approx 3,4551 \cdot 10^{-4}A} \\[15pt]
        $(1): i_T -i_2 - i_3 = 0 \Rightarrow i_3 = i_T - i_2$ \\[10pt]
        $i_3 = \frac{8}{9551}A - \frac{33}{95510} \cdot 10^{-4}$ \\[10pt]
        \resaltar{i_3 = \frac{47}{95510}A \approx 4,9209 \cdot 10^{-4}A}
    \end{center}

    \saltoPag{}
    \subsection{Resumen de los resultados}
    \begin{center}
        \small
        \begin{tabular}{| c c c c c |}
            \hline
            \multicolumn{1}{|m{1.2cm}}{\centering\textbf{Valores}} &
            \multicolumn{1}{m{1.2cm}}{\centering\textbf{$V_S$}} &
            \multicolumn{1}{m{1.2cm}}{\centering\textbf{$R_1$}} &
            \multicolumn{1}{m{1.2cm}}{\centering\textbf{$R_2$}} &
            \multicolumn{1}{m{1.2cm}|}{\centering\textbf{$R_3$}} \\
            \hline
            \multicolumn{1}{|m{1.2cm}|}{\centering\textbf{Voltaje}} &
            \multicolumn{1}{m{1.2cm}}{\centering$10V$} &
            \multicolumn{1}{m{1.2cm}}{\centering$\frac{8\cdot 10^{4}}{9551}V$} &
            \multicolumn{1}{m{1.2cm}}{\centering$\frac{15510}{9551}V$} &
            \multicolumn{1}{m{1.2cm}|}{\centering$\frac{15510}{9551}V$} \\
            \multicolumn{1}{|m{1.2cm}|}{\centering\textbf{Corriente}} &
            \multicolumn{1}{m{1.2cm}}{\centering$\frac{8}{9551}A^*$} &
            \multicolumn{1}{m{1.2cm}}{\centering$\frac{8}{9551}A$} &
            \multicolumn{1}{m{1.2cm}}{\centering$\frac{33}{95510}A$} &
            \multicolumn{1}{m{1.2cm}|}{\centering$\frac{47}{95510}A$} \\
            \hline
        \end{tabular} 
    \end{center}
   \midTitle{black}{Resumen de los resultados aproximados}
    \begin{center}
        \footnotesize
        \begin{tabular}{| c c c c c |}
            \hline
            \multicolumn{1}{|m{1.25cm}}{\centering\textbf{Valores}} &
            \multicolumn{1}{m{1.25cm}}{\centering\textbf{$V_S$}} &
            \multicolumn{1}{m{1.25cm}}{\centering\textbf{$R_1$}} &
            \multicolumn{1}{m{1.25cm}}{\centering\textbf{$R_2$}} &
            \multicolumn{1}{m{1.25cm}|}{\centering\textbf{$R_3$}} \\
            \hline
            \multicolumn{1}{|m{1.25cm}|}{\centering\textbf{Voltaje}} &
            \multicolumn{1}{m{1.25cm}}{\centering$10V$} &
            \multicolumn{1}{m{1.25cm}}{\centering$8,3760V$} &
            \multicolumn{1}{m{1.25cm}}{\centering$1,6239V$} &
            \multicolumn{1}{m{1.25cm}|}{\centering$1,6239V$} \\
            \multicolumn{1}{|m{1.25cm}|}{\centering\textbf{Corriente}} &
            \multicolumn{1}{m{1.25cm}}{\centering$0,83760 mA^*$} &
            \multicolumn{1}{m{1.25cm}}{\centering$0,83760  mA$} &
            \multicolumn{1}{m{1.25cm}}{\centering$0,34551  mA$} &
            \multicolumn{1}{m{1.25cm}|}{\centering$0,4920 mA$} \\
            \hline
        \end{tabular} 
    \end{center}
   *: Siendo en este análisis una fuente ideal, se estima que la fuente entregará la corriente necesaria que consuma el circuito (dentro de lo físicamente razonable), manteniendo su voltaje constante. En la realidad, esto no sucede.

