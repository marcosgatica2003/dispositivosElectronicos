% !TEX root = main.tex
\section{Actividad práctica II}
\sangria{} Se utilizará el simulador LTspice para corroborar los cálculos realizados en la actividad práctica I.
\begin{center}
    \begin{circuitikz}[raised voltages]
        \draw (0,0)
         to[battery, v=$V_ S$, invert, f>_=$I_T$] (0,3)
         to[R=$R_1$, f>_=$I_1$] (3,3)
         to[short, *-*] (3,3)
         to[R=$R_2$, f>_=$I_2$] (3,0) -- (0,0);

         \draw (3,3) -- (6,3)
         to[R=$R_3$, f>_=$I_3$] (6,0) -- (3,0)
         to[short, *-*] (3,0);
        
         \node at (0,3.2) {A};
         \node at (3,3.2) {B};
         \node at (0,-0.2) {C};
    \end{circuitikz}
    \end{center}
    
\midTitle{blue}{Información del instrumento de medición} \begin{itemize} \item \textbf{Fabricante:} Pro'sKit \item \textbf{Modelo:} MT-1706 \item \textbf{Serie:} CAT 1000V  \end{itemize} \imagen{6cm}{./imagenes/multimetro.jpg} \begin{itemize} \item \textbf{Temperatura del ambiente al medir:} $20^oC$ \imagen{5cm}{./imagenes/temperaturaAmbiente.jpg}\end{itemize} 
    \begin{center} 
        {\small
        \begin{tabular}{| p{1cm}  p{1cm}  p{1cm} |}
            \hline 
            \multicolumn{3}{|c|}{\textbf{Información de mediciones}} \\
            \hline
            \multicolumn{1}{|m{2cm}}{\centering\textbf{Tensión CC}} &
            \multicolumn{1}{m{2cm}}{\centering\textbf{Temperatura $[^oC]$}} &
            \multicolumn{1}{m{2cm}|}{\centering\textbf{Resistencia}} \\
            \hline
            \multicolumn{1}{|m{2cm}}{$\pm0,5\%$ lectura + $3$ dig.} &
            \multicolumn{1}{m{2cm}}{$\pm1,0\% lectura$ + $3$} &
            \multicolumn{1}{m{2cm}|}{$\pm 0,8\%$ lectura + $3$ dig.} \\
            \hline

    \end{tabular}}
    \end{center}

\midTitle{blue}{Valores reales del circuito}
        \begin{center} 
            \begin{tabular}{| p{1cm}  p{1cm}  p{1cm} |} 
                \hline 
                \multicolumn{1}{|m{1.4cm}}{\centering\textbf{Resistores}} &
                \multicolumn{1}{m{1.4cm}}{\centering\textbf{Valor Nominal [$\Omega$]}} &
                \multicolumn{1}{m{1.4cm}|}{\centering\textbf{Valor Real [$\Omega$]}} \\
                \hline
                \multicolumn{1}{|c}{$R_1$} &
                \multicolumn{1}{c}{$10K\Omega$} &
                \multicolumn{1}{c|}{$10,04k\Omega$} \\
                \multicolumn{1}{|c}{$R_2$} &
                \multicolumn{1}{c}{$4,7K\Omega$} &
                \multicolumn{1}{c|}{$4,68k\Omega$} \\
                \multicolumn{1}{|c}{$R_3$} &
                \multicolumn{1}{c}{$3,3K\Omega$} &
                \multicolumn{1}{c|}{$3,23k\Omega$} \\
                \hline
            \end{tabular}
        \end{center}
        \sangria{} En general, los resistores se midieron de la siguiente forma: aislados del circuito conectando las puntas a sus terminales, en la escala de $60k\Omega$. \imagen{4cm}{./imagenes/valoresRealesResistencias.jpg}
        \sangria{} Tensión CC de la fuente: \imagen{4cm}{./imagenes/tensionReal.jpg} \saltoPag{}
\midTitle{red}{Información del software y el equipo}
\begin{itemize} \item \textbf{Sistema base:} GNU/Ubuntu 24.04 x86 x64\item \textbf{Versión de LTspice:} 24.1.6 x Wine\item \textbf{Persona a cargo:} el documentador. \end{itemize}
\sangria{} LTspice es un software gratuito de simulación SPICE (\textit{Simulation Program with Integrated Circuits Emphasis}) propiedad del fabricante de circuitos integrados fabricante. \\ \sangria{} Su elección fue debido a su facilidad de instalación y consumo de recursos, además de que es un software recomendado por la cátedra de la asignatura.
\midTitle{red}{\textbf{Circuito en el LTspice}} \imagen{8cm}{./imagenes/circuitoSpice.png}
\midTitle{red}{\textbf{Output del simulador}} \imagen{8cm}{./imagenes/outputSpice.png}
\noindent
